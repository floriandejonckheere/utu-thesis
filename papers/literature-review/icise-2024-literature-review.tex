\documentclass[sigconf]{acmart}

\AtBeginDocument{\providecommand\BibTeX{{Bib\TeX}}}

\copyrightyear{2024}
\acmYear{2024}
\setcopyright{rightsretained}
\acmConference[ICISE 2024]{2024 9th International Conference on Information Systems Engineering}{December 16--18, 2024}{Chiang Mai, Thailand}
\acmBooktitle{2024 9th International Conference on Information Systems Engineering (ICISE 2024), December 16--18, 2024, Chiang Mai, Thailand}
\acmDOI{10.1145/3641032.3641042}
\acmISBN{979-8-4007-0917-3/23/12}

\begin{document}

	\title{A Survey on Modularization and Microservice Candidate Identification in Monolith Systems}

	\author{Florian Dejonckheere}
	\email{florian@floriandejonckheere.be}
	\affiliation{
		\institution{}
		\city{}
		\country{}
	}

	\author{Sampsa Rauti}
	\email{sjprau@utu.fi}
	\affiliation{
		\institution{University of Turku}
		\city{Turku}
		\country{Finland}
	}

	\author{Tuomas Mäkilä}
	\email{tusuma@utu.fi}
	\affiliation{
		\institution{University of Turku}
		\city{Turku}
		\country{Finland}
	}

	\begin{abstract}
		This paper dives into the current state of (semi-)automated technologies for modularizing monolithic codebases, with a particular focus in microservices candidate identification.
		Identification of suitable microservices is a key aspect in transitioning from monoliths to microservices architecture.
		We conducted a systematic literature review, categorizing existing approaches and techniques that emphasize automation, and discuss the challenges and opportunities in this area.
		The literature review identified 43 approaches, which we categorized by the type of input used for the identification process, the class(es) of algorithms, and the quality metrics used to evaluate the decomposition.
		We found that the majority of approaches use static analysis of software lifecycle development artifacts, and that the most common algorithms are based on graph theory.
		We also found that the most common quality metrics are based on coupling and cohesion.
		Finally, we discuss the implications of these findings for software development and suggest future research directions.
	\end{abstract}

	\begin{CCSXML}
		<ccs2012>
		<concept>
		<concept_id>10002978.10003022.10003026</concept_id>
		<concept_desc>Security and privacy~Web application security</concept_desc>
		<concept_significance>500</concept_significance>
		</concept>
		</ccs2012>
	\end{CCSXML}

	\ccsdesc[500]{Security and privacy~Web application security}

	\keywords{Software architecture, monolithic architecture, microservice architecture, microservice candidate identification}

	\maketitle

	\section{Introduction}

	...

	\section{Study Setting and Method}

	...

	In this study, we conducted a systematic literature review (SLR), as presented by Kitchenham et al. \cite{kitchenham2004}.

	- used search terms

	- criteria for including and excluding papers etc.

	\section{Results}

	...

	- some meta info, e.g. a bar chart of the publications years

	- discuss the papers, maybe in some kinds of thematic categories

	- synthesis

	\section{Discussion}

	\subsection{Key Findings}

	...

	\subsection{Implications for Software Development}

	...

	\subsection{Limitations and Future Research}

	...

	\section{Conclusions}

	...

	%\begin{acks}

	%...

	%\end{acks}

	\bibliographystyle{ACM-Reference-Format}
	\bibliography{icise-2024-literature-review}

\end{document}
\endinput
