% Document template suitable for use as a LaTeX master-file 
% for thesis works in University of Turku Department of Computing
%
% Technical usage guide: https://tech.utugit.fi/soft/thesis/doc/doc/overview/
% 

\documentclass[language=finnish,version=final,mainfont=none,sharelatex=false]{utuftthesis}
\setcounter{secnumdepth}{2}
\setcounter{tocdepth}{2}
\usepackage{float}
\usepackage[caption=false]{subfig}

% Define the algorithm environment
%\makeatletter
\providecommand\textquotedblplain{%
  \bgroup\addfontfeatures{Mapping=}\char34\egroup}
\providecommand{\tabularnewline}{\\}
\floatstyle{ruled}
\newfloat{algorithm}{tbp}{loa}
\providecommand{\algorithmname}{Algoritmi}
\floatname{algorithm}{\protect\algorithmname}
%\makeatother

\addbibresource{thesis.bib}

\begin{document}

\pubyear{2019}
\pubmonth{6}
\publab{Labran nimi}
\publaben{Laboratory Name}
\pubtype{tkk}
\title{Name of Thesis}
\author{My Name}

\maketitle
\keywords{here, a, list, of, keywords}
\keywordsen{here, a, list, of, keywords}

\begin{abstract}
\end{abstract}

\begin{abstracten}
\end{abstracten}



% mandatory
\tableofcontents

% if you want a list of figures
\listoffigures

% if you want a list of tables
\listoftables

% if you want a list of acronyms
\listofacronyms

% change the name if the default doesn't sound right
\renewcommand{\algorithmname}{\listingscaption}

% The thesis starts here.

\begin{comment}
To better organize things, create a new tex file for each chapter
and input it below.

Avoid using the å, ä, ö or <space> characters in referred names and
underscores \_ in file names (may break hyperref).

Good luck!
\end{comment}

%\input{file_name_of_chapter_x}
%\input{file_name_of_chapter_y}

% The thesis main content ends here.

\printbibliography

\begin{comment}
Important! Create the appendix chapters with command \textbackslash appchapter\{some
name\} instead of \textbackslash chapter\{some name\} for the automagic
page counting to work!
\end{comment}


\appchapter{Liitedokumentti}

Liitteen ohjelmakoodi \ref{alg:Tyyppiluokka-Monad} kuvaa matemaattisen
monadirakenteen pohjalta rakentuvan Haskellin tyyppiluokan. Tyyppiluokan
voi nähdä eräänlaisena abstraktina ohjelmointirajapintana (API\nomenclature[API]{API}{Application Programming Interface}),
joka muodostaa ohjelmoijalle abstraktin ohjelmointikielen käyttöliittymän
(UI\nomenclature[UI]{UI}{User Interface}).

\begin{algorithm}[tbh]
\begin{minted}{haskell}
class Monad m where
    ( >>= )         :: m a -> (a -> m b) -> m b
    return          :: a                 -> m a

    fail            :: String            -> m a
    (>>)            :: m a -> m b        -> m b
    m >> k          =  m >>= \_ -> k       -- default

instance Monad IO where  ...               -- omitted
\end{minted}

\caption{Tyyppiluokka 'Monad'.\label{alg:Tyyppiluokka-Monad}}
\end{algorithm}

\newpage{}

Ensimmäisen liitteen toinen sivu. Ohjelmalistaus \ref{alg:Monadin-kayttoa}
demonstroi vielä monadin käyttöä.

\begin{algorithm}[tbh]
\begin{minted}{haskell}
main =
return "Your name:" >>=
putStr >>=
\_ -> getLine >>=
\n -> putStrLn ("Hey " ++ n)
\end{minted}

\caption{Monadin käyttöä.\label{alg:Monadin-kayttoa}}
\end{algorithm}


\appchapter{Liitedokumentti 2}

Tässä esimerkki\pagebreak{}

toisesta kaksisivuisesta liitteestä.
\end{document}
