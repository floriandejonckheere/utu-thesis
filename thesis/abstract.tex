
\keywords{tähän, lista, avainsanoista}
% TODO: good/bad keywords

\keywordsen{here, a, list, of, keywords}
\begin{abstract}
Tarkempia ohjeita tiivistelmäsivun laadintaan läytyy opiskelijan yleisoppaasta,
josta alla lyhyt katkelma.

Bibliografisten tietojen jälkeen kirjoitetaan varsinainen tiivistelmä.
Sen on oletetta\-va, että lukijalla on yleiset tiedot aiheesta. Tiivistelmän
tulee olla ymmärrettävissä ilman tarvetta perehtyä koko tutkielmaan.
Se on kirjoitettava täydellisinä virkkeinä, väliotsakeluettelona.
On käytettävä vakiintuneita termejä. Viittauksia ja lainauksia tiivistelmään
ei saa sisällyttää, eikä myäskään tietoja tai väitteitä, jotka eivät
sisälly itse tutkimukseen. Tiivistelmän on oltava mahdollisimman ytimekäs
n. 120–250 sanan pituinen itsenäinen kokonaisuus, joka mahtuu ykkösvälillä
kirjoitettuna vaivatta yhdelle tiivistelmäsivulle. Tiivistelmässä tulisi ilmetä
mm.  tutkielman aihe tutkimuksen kohde, populaatio, alue ja tarkoitus
käytetyt tutkimusmenetelmät (mikäli tutkimus on luonteeltaan teoreettinen
ja tiettyyn kirjalliseen materiaaliin, on mainittava tärkeimmät lähdeteokset;
mikäli on luonteeltaan empiirinen, on mainittava käytetyt metodit)
keskeiset tutkimustulokset tulosten perusteella tehdyt päätelmät ja
toimenpidesuositukset.
\end{abstract}

\begin{abstracten}
Second abstract in english (in case the document main language is
not english)
\end{abstracten}

